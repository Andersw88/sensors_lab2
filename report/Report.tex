\documentclass[DIV12,a4paper]{scrartcl}

\usepackage%[demo]%
{graphicx}
\usepackage[utf8]{inputenc}
\usepackage{listings}
\usepackage{caption}
\usepackage{subcaption}

\graphicspath{ {figures/} }

\newcommand{\defaultwidth}{.8\textwidth}
\newcommand{\halfwidth}{.4\textwidth}


\title{Laboration 2:\\ ASUS Xtion Pro: Calibration, noise characterization and filtering\\{\small Sensors and Sensing}}
\author{Michael Flo{\ss}mann, Anders Wilkstr\"om}
\date{2015--12--07}

\begin{document}
\maketitle

\section{Introduction: Structured light cameras}
Structured light cameras are a low-cost option for depth measuring in three dimensional space. The cameras project a known light pattern to a scene and record the reflection of that light pattern. This recorded data is then used for triangulation.\par
For this lab, the ASUS Xtion Pro sensor was used as a structured light camera.
\section{Task and implementation}
The task at hand was to set up and calibrating the sensor, as well as to characterize the noise in the depth measurement and to set up filtering routines.
\subsection{Basic setup}
To set up the camera, the package \texttt{openni2} for ros-indigo was used. When launching the node \texttt{openni2.launch}, it publishes a wide range of topics from the camera.\par %TODO: Maybe list ALL the topics?
For this laboration, only the topics which publish a viewable image were of interest. This included two main topics:%TODO: Explain the opencv shiat

\begin{itemize}
  \item \texttt{/camera/rgb/}\\
    This topic publishes data from the RGB camera on the ASUS Xtion Pro. The topic \texttt{/camera/rgb/raw} shows the unprocessed RGB image like a regular camera. A sample image from this topic is shown in figure %\ref{} TODO!
  \item \texttt{/camera/depth/}\\
    This topic publishes the depth data as a 2D-array of float variables with the depth values in meters.
\end{itemize}

\begin{figure}
  \centering
  \includegraphics[width=\defaultwidth]{narf.png}
  \label{fig:rgb-raw}
  \caption{Output image of \texttt{/camera/rgb/raw}}
\end{figure}

\end{document}
